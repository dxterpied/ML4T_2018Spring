
% Default to the notebook output style

    


% Inherit from the specified cell style.




    
\documentclass[11pt]{article}

    
    
    \usepackage[T1]{fontenc}
    % Nicer default font (+ math font) than Computer Modern for most use cases
    \usepackage{mathpazo}

    % Basic figure setup, for now with no caption control since it's done
    % automatically by Pandoc (which extracts ![](path) syntax from Markdown).
    \usepackage{graphicx}
    % We will generate all images so they have a width \maxwidth. This means
    % that they will get their normal width if they fit onto the page, but
    % are scaled down if they would overflow the margins.
    \makeatletter
    \def\maxwidth{\ifdim\Gin@nat@width>\linewidth\linewidth
    \else\Gin@nat@width\fi}
    \makeatother
    \let\Oldincludegraphics\includegraphics
    % Set max figure width to be 80% of text width, for now hardcoded.
    \renewcommand{\includegraphics}[1]{\Oldincludegraphics[width=.8\maxwidth]{#1}}
    % Ensure that by default, figures have no caption (until we provide a
    % proper Figure object with a Caption API and a way to capture that
    % in the conversion process - todo).
    \usepackage{caption}
    \DeclareCaptionLabelFormat{nolabel}{}
    \captionsetup{labelformat=nolabel}

    \usepackage{adjustbox} % Used to constrain images to a maximum size 
    \usepackage{xcolor} % Allow colors to be defined
    \usepackage{enumerate} % Needed for markdown enumerations to work
    \usepackage{geometry} % Used to adjust the document margins
    \usepackage{amsmath} % Equations
    \usepackage{amssymb} % Equations
    \usepackage{textcomp} % defines textquotesingle
    % Hack from http://tex.stackexchange.com/a/47451/13684:
    \AtBeginDocument{%
        \def\PYZsq{\textquotesingle}% Upright quotes in Pygmentized code
    }
    \usepackage{upquote} % Upright quotes for verbatim code
    \usepackage{eurosym} % defines \euro
    \usepackage[mathletters]{ucs} % Extended unicode (utf-8) support
    \usepackage[utf8x]{inputenc} % Allow utf-8 characters in the tex document
    \usepackage{fancyvrb} % verbatim replacement that allows latex
    \usepackage{grffile} % extends the file name processing of package graphics 
                         % to support a larger range 
    % The hyperref package gives us a pdf with properly built
    % internal navigation ('pdf bookmarks' for the table of contents,
    % internal cross-reference links, web links for URLs, etc.)
    \usepackage{hyperref}
    \usepackage{longtable} % longtable support required by pandoc >1.10
    \usepackage{booktabs}  % table support for pandoc > 1.12.2
    \usepackage[inline]{enumitem} % IRkernel/repr support (it uses the enumerate* environment)
    \usepackage[normalem]{ulem} % ulem is needed to support strikethroughs (\sout)
                                % normalem makes italics be italics, not underlines
    

    
    
    % Colors for the hyperref package
    \definecolor{urlcolor}{rgb}{0,.145,.698}
    \definecolor{linkcolor}{rgb}{.71,0.21,0.01}
    \definecolor{citecolor}{rgb}{.12,.54,.11}

    % ANSI colors
    \definecolor{ansi-black}{HTML}{3E424D}
    \definecolor{ansi-black-intense}{HTML}{282C36}
    \definecolor{ansi-red}{HTML}{E75C58}
    \definecolor{ansi-red-intense}{HTML}{B22B31}
    \definecolor{ansi-green}{HTML}{00A250}
    \definecolor{ansi-green-intense}{HTML}{007427}
    \definecolor{ansi-yellow}{HTML}{DDB62B}
    \definecolor{ansi-yellow-intense}{HTML}{B27D12}
    \definecolor{ansi-blue}{HTML}{208FFB}
    \definecolor{ansi-blue-intense}{HTML}{0065CA}
    \definecolor{ansi-magenta}{HTML}{D160C4}
    \definecolor{ansi-magenta-intense}{HTML}{A03196}
    \definecolor{ansi-cyan}{HTML}{60C6C8}
    \definecolor{ansi-cyan-intense}{HTML}{258F8F}
    \definecolor{ansi-white}{HTML}{C5C1B4}
    \definecolor{ansi-white-intense}{HTML}{A1A6B2}

    % commands and environments needed by pandoc snippets
    % extracted from the output of `pandoc -s`
    \providecommand{\tightlist}{%
      \setlength{\itemsep}{0pt}\setlength{\parskip}{0pt}}
    \DefineVerbatimEnvironment{Highlighting}{Verbatim}{commandchars=\\\{\}}
    % Add ',fontsize=\small' for more characters per line
    \newenvironment{Shaded}{}{}
    \newcommand{\KeywordTok}[1]{\textcolor[rgb]{0.00,0.44,0.13}{\textbf{{#1}}}}
    \newcommand{\DataTypeTok}[1]{\textcolor[rgb]{0.56,0.13,0.00}{{#1}}}
    \newcommand{\DecValTok}[1]{\textcolor[rgb]{0.25,0.63,0.44}{{#1}}}
    \newcommand{\BaseNTok}[1]{\textcolor[rgb]{0.25,0.63,0.44}{{#1}}}
    \newcommand{\FloatTok}[1]{\textcolor[rgb]{0.25,0.63,0.44}{{#1}}}
    \newcommand{\CharTok}[1]{\textcolor[rgb]{0.25,0.44,0.63}{{#1}}}
    \newcommand{\StringTok}[1]{\textcolor[rgb]{0.25,0.44,0.63}{{#1}}}
    \newcommand{\CommentTok}[1]{\textcolor[rgb]{0.38,0.63,0.69}{\textit{{#1}}}}
    \newcommand{\OtherTok}[1]{\textcolor[rgb]{0.00,0.44,0.13}{{#1}}}
    \newcommand{\AlertTok}[1]{\textcolor[rgb]{1.00,0.00,0.00}{\textbf{{#1}}}}
    \newcommand{\FunctionTok}[1]{\textcolor[rgb]{0.02,0.16,0.49}{{#1}}}
    \newcommand{\RegionMarkerTok}[1]{{#1}}
    \newcommand{\ErrorTok}[1]{\textcolor[rgb]{1.00,0.00,0.00}{\textbf{{#1}}}}
    \newcommand{\NormalTok}[1]{{#1}}
    
    % Additional commands for more recent versions of Pandoc
    \newcommand{\ConstantTok}[1]{\textcolor[rgb]{0.53,0.00,0.00}{{#1}}}
    \newcommand{\SpecialCharTok}[1]{\textcolor[rgb]{0.25,0.44,0.63}{{#1}}}
    \newcommand{\VerbatimStringTok}[1]{\textcolor[rgb]{0.25,0.44,0.63}{{#1}}}
    \newcommand{\SpecialStringTok}[1]{\textcolor[rgb]{0.73,0.40,0.53}{{#1}}}
    \newcommand{\ImportTok}[1]{{#1}}
    \newcommand{\DocumentationTok}[1]{\textcolor[rgb]{0.73,0.13,0.13}{\textit{{#1}}}}
    \newcommand{\AnnotationTok}[1]{\textcolor[rgb]{0.38,0.63,0.69}{\textbf{\textit{{#1}}}}}
    \newcommand{\CommentVarTok}[1]{\textcolor[rgb]{0.38,0.63,0.69}{\textbf{\textit{{#1}}}}}
    \newcommand{\VariableTok}[1]{\textcolor[rgb]{0.10,0.09,0.49}{{#1}}}
    \newcommand{\ControlFlowTok}[1]{\textcolor[rgb]{0.00,0.44,0.13}{\textbf{{#1}}}}
    \newcommand{\OperatorTok}[1]{\textcolor[rgb]{0.40,0.40,0.40}{{#1}}}
    \newcommand{\BuiltInTok}[1]{{#1}}
    \newcommand{\ExtensionTok}[1]{{#1}}
    \newcommand{\PreprocessorTok}[1]{\textcolor[rgb]{0.74,0.48,0.00}{{#1}}}
    \newcommand{\AttributeTok}[1]{\textcolor[rgb]{0.49,0.56,0.16}{{#1}}}
    \newcommand{\InformationTok}[1]{\textcolor[rgb]{0.38,0.63,0.69}{\textbf{\textit{{#1}}}}}
    \newcommand{\WarningTok}[1]{\textcolor[rgb]{0.38,0.63,0.69}{\textbf{\textit{{#1}}}}}
    
    
    % Define a nice break command that doesn't care if a line doesn't already
    % exist.
    \def\br{\hspace*{\fill} \\* }
    % Math Jax compatability definitions
    \def\gt{>}
    \def\lt{<}
    % Document parameters
    \title{strategy\_learner}
    
    
    

    % Pygments definitions
    
\makeatletter
\def\PY@reset{\let\PY@it=\relax \let\PY@bf=\relax%
    \let\PY@ul=\relax \let\PY@tc=\relax%
    \let\PY@bc=\relax \let\PY@ff=\relax}
\def\PY@tok#1{\csname PY@tok@#1\endcsname}
\def\PY@toks#1+{\ifx\relax#1\empty\else%
    \PY@tok{#1}\expandafter\PY@toks\fi}
\def\PY@do#1{\PY@bc{\PY@tc{\PY@ul{%
    \PY@it{\PY@bf{\PY@ff{#1}}}}}}}
\def\PY#1#2{\PY@reset\PY@toks#1+\relax+\PY@do{#2}}

\expandafter\def\csname PY@tok@w\endcsname{\def\PY@tc##1{\textcolor[rgb]{0.73,0.73,0.73}{##1}}}
\expandafter\def\csname PY@tok@c\endcsname{\let\PY@it=\textit\def\PY@tc##1{\textcolor[rgb]{0.25,0.50,0.50}{##1}}}
\expandafter\def\csname PY@tok@cp\endcsname{\def\PY@tc##1{\textcolor[rgb]{0.74,0.48,0.00}{##1}}}
\expandafter\def\csname PY@tok@k\endcsname{\let\PY@bf=\textbf\def\PY@tc##1{\textcolor[rgb]{0.00,0.50,0.00}{##1}}}
\expandafter\def\csname PY@tok@kp\endcsname{\def\PY@tc##1{\textcolor[rgb]{0.00,0.50,0.00}{##1}}}
\expandafter\def\csname PY@tok@kt\endcsname{\def\PY@tc##1{\textcolor[rgb]{0.69,0.00,0.25}{##1}}}
\expandafter\def\csname PY@tok@o\endcsname{\def\PY@tc##1{\textcolor[rgb]{0.40,0.40,0.40}{##1}}}
\expandafter\def\csname PY@tok@ow\endcsname{\let\PY@bf=\textbf\def\PY@tc##1{\textcolor[rgb]{0.67,0.13,1.00}{##1}}}
\expandafter\def\csname PY@tok@nb\endcsname{\def\PY@tc##1{\textcolor[rgb]{0.00,0.50,0.00}{##1}}}
\expandafter\def\csname PY@tok@nf\endcsname{\def\PY@tc##1{\textcolor[rgb]{0.00,0.00,1.00}{##1}}}
\expandafter\def\csname PY@tok@nc\endcsname{\let\PY@bf=\textbf\def\PY@tc##1{\textcolor[rgb]{0.00,0.00,1.00}{##1}}}
\expandafter\def\csname PY@tok@nn\endcsname{\let\PY@bf=\textbf\def\PY@tc##1{\textcolor[rgb]{0.00,0.00,1.00}{##1}}}
\expandafter\def\csname PY@tok@ne\endcsname{\let\PY@bf=\textbf\def\PY@tc##1{\textcolor[rgb]{0.82,0.25,0.23}{##1}}}
\expandafter\def\csname PY@tok@nv\endcsname{\def\PY@tc##1{\textcolor[rgb]{0.10,0.09,0.49}{##1}}}
\expandafter\def\csname PY@tok@no\endcsname{\def\PY@tc##1{\textcolor[rgb]{0.53,0.00,0.00}{##1}}}
\expandafter\def\csname PY@tok@nl\endcsname{\def\PY@tc##1{\textcolor[rgb]{0.63,0.63,0.00}{##1}}}
\expandafter\def\csname PY@tok@ni\endcsname{\let\PY@bf=\textbf\def\PY@tc##1{\textcolor[rgb]{0.60,0.60,0.60}{##1}}}
\expandafter\def\csname PY@tok@na\endcsname{\def\PY@tc##1{\textcolor[rgb]{0.49,0.56,0.16}{##1}}}
\expandafter\def\csname PY@tok@nt\endcsname{\let\PY@bf=\textbf\def\PY@tc##1{\textcolor[rgb]{0.00,0.50,0.00}{##1}}}
\expandafter\def\csname PY@tok@nd\endcsname{\def\PY@tc##1{\textcolor[rgb]{0.67,0.13,1.00}{##1}}}
\expandafter\def\csname PY@tok@s\endcsname{\def\PY@tc##1{\textcolor[rgb]{0.73,0.13,0.13}{##1}}}
\expandafter\def\csname PY@tok@sd\endcsname{\let\PY@it=\textit\def\PY@tc##1{\textcolor[rgb]{0.73,0.13,0.13}{##1}}}
\expandafter\def\csname PY@tok@si\endcsname{\let\PY@bf=\textbf\def\PY@tc##1{\textcolor[rgb]{0.73,0.40,0.53}{##1}}}
\expandafter\def\csname PY@tok@se\endcsname{\let\PY@bf=\textbf\def\PY@tc##1{\textcolor[rgb]{0.73,0.40,0.13}{##1}}}
\expandafter\def\csname PY@tok@sr\endcsname{\def\PY@tc##1{\textcolor[rgb]{0.73,0.40,0.53}{##1}}}
\expandafter\def\csname PY@tok@ss\endcsname{\def\PY@tc##1{\textcolor[rgb]{0.10,0.09,0.49}{##1}}}
\expandafter\def\csname PY@tok@sx\endcsname{\def\PY@tc##1{\textcolor[rgb]{0.00,0.50,0.00}{##1}}}
\expandafter\def\csname PY@tok@m\endcsname{\def\PY@tc##1{\textcolor[rgb]{0.40,0.40,0.40}{##1}}}
\expandafter\def\csname PY@tok@gh\endcsname{\let\PY@bf=\textbf\def\PY@tc##1{\textcolor[rgb]{0.00,0.00,0.50}{##1}}}
\expandafter\def\csname PY@tok@gu\endcsname{\let\PY@bf=\textbf\def\PY@tc##1{\textcolor[rgb]{0.50,0.00,0.50}{##1}}}
\expandafter\def\csname PY@tok@gd\endcsname{\def\PY@tc##1{\textcolor[rgb]{0.63,0.00,0.00}{##1}}}
\expandafter\def\csname PY@tok@gi\endcsname{\def\PY@tc##1{\textcolor[rgb]{0.00,0.63,0.00}{##1}}}
\expandafter\def\csname PY@tok@gr\endcsname{\def\PY@tc##1{\textcolor[rgb]{1.00,0.00,0.00}{##1}}}
\expandafter\def\csname PY@tok@ge\endcsname{\let\PY@it=\textit}
\expandafter\def\csname PY@tok@gs\endcsname{\let\PY@bf=\textbf}
\expandafter\def\csname PY@tok@gp\endcsname{\let\PY@bf=\textbf\def\PY@tc##1{\textcolor[rgb]{0.00,0.00,0.50}{##1}}}
\expandafter\def\csname PY@tok@go\endcsname{\def\PY@tc##1{\textcolor[rgb]{0.53,0.53,0.53}{##1}}}
\expandafter\def\csname PY@tok@gt\endcsname{\def\PY@tc##1{\textcolor[rgb]{0.00,0.27,0.87}{##1}}}
\expandafter\def\csname PY@tok@err\endcsname{\def\PY@bc##1{\setlength{\fboxsep}{0pt}\fcolorbox[rgb]{1.00,0.00,0.00}{1,1,1}{\strut ##1}}}
\expandafter\def\csname PY@tok@kc\endcsname{\let\PY@bf=\textbf\def\PY@tc##1{\textcolor[rgb]{0.00,0.50,0.00}{##1}}}
\expandafter\def\csname PY@tok@kd\endcsname{\let\PY@bf=\textbf\def\PY@tc##1{\textcolor[rgb]{0.00,0.50,0.00}{##1}}}
\expandafter\def\csname PY@tok@kn\endcsname{\let\PY@bf=\textbf\def\PY@tc##1{\textcolor[rgb]{0.00,0.50,0.00}{##1}}}
\expandafter\def\csname PY@tok@kr\endcsname{\let\PY@bf=\textbf\def\PY@tc##1{\textcolor[rgb]{0.00,0.50,0.00}{##1}}}
\expandafter\def\csname PY@tok@bp\endcsname{\def\PY@tc##1{\textcolor[rgb]{0.00,0.50,0.00}{##1}}}
\expandafter\def\csname PY@tok@fm\endcsname{\def\PY@tc##1{\textcolor[rgb]{0.00,0.00,1.00}{##1}}}
\expandafter\def\csname PY@tok@vc\endcsname{\def\PY@tc##1{\textcolor[rgb]{0.10,0.09,0.49}{##1}}}
\expandafter\def\csname PY@tok@vg\endcsname{\def\PY@tc##1{\textcolor[rgb]{0.10,0.09,0.49}{##1}}}
\expandafter\def\csname PY@tok@vi\endcsname{\def\PY@tc##1{\textcolor[rgb]{0.10,0.09,0.49}{##1}}}
\expandafter\def\csname PY@tok@vm\endcsname{\def\PY@tc##1{\textcolor[rgb]{0.10,0.09,0.49}{##1}}}
\expandafter\def\csname PY@tok@sa\endcsname{\def\PY@tc##1{\textcolor[rgb]{0.73,0.13,0.13}{##1}}}
\expandafter\def\csname PY@tok@sb\endcsname{\def\PY@tc##1{\textcolor[rgb]{0.73,0.13,0.13}{##1}}}
\expandafter\def\csname PY@tok@sc\endcsname{\def\PY@tc##1{\textcolor[rgb]{0.73,0.13,0.13}{##1}}}
\expandafter\def\csname PY@tok@dl\endcsname{\def\PY@tc##1{\textcolor[rgb]{0.73,0.13,0.13}{##1}}}
\expandafter\def\csname PY@tok@s2\endcsname{\def\PY@tc##1{\textcolor[rgb]{0.73,0.13,0.13}{##1}}}
\expandafter\def\csname PY@tok@sh\endcsname{\def\PY@tc##1{\textcolor[rgb]{0.73,0.13,0.13}{##1}}}
\expandafter\def\csname PY@tok@s1\endcsname{\def\PY@tc##1{\textcolor[rgb]{0.73,0.13,0.13}{##1}}}
\expandafter\def\csname PY@tok@mb\endcsname{\def\PY@tc##1{\textcolor[rgb]{0.40,0.40,0.40}{##1}}}
\expandafter\def\csname PY@tok@mf\endcsname{\def\PY@tc##1{\textcolor[rgb]{0.40,0.40,0.40}{##1}}}
\expandafter\def\csname PY@tok@mh\endcsname{\def\PY@tc##1{\textcolor[rgb]{0.40,0.40,0.40}{##1}}}
\expandafter\def\csname PY@tok@mi\endcsname{\def\PY@tc##1{\textcolor[rgb]{0.40,0.40,0.40}{##1}}}
\expandafter\def\csname PY@tok@il\endcsname{\def\PY@tc##1{\textcolor[rgb]{0.40,0.40,0.40}{##1}}}
\expandafter\def\csname PY@tok@mo\endcsname{\def\PY@tc##1{\textcolor[rgb]{0.40,0.40,0.40}{##1}}}
\expandafter\def\csname PY@tok@ch\endcsname{\let\PY@it=\textit\def\PY@tc##1{\textcolor[rgb]{0.25,0.50,0.50}{##1}}}
\expandafter\def\csname PY@tok@cm\endcsname{\let\PY@it=\textit\def\PY@tc##1{\textcolor[rgb]{0.25,0.50,0.50}{##1}}}
\expandafter\def\csname PY@tok@cpf\endcsname{\let\PY@it=\textit\def\PY@tc##1{\textcolor[rgb]{0.25,0.50,0.50}{##1}}}
\expandafter\def\csname PY@tok@c1\endcsname{\let\PY@it=\textit\def\PY@tc##1{\textcolor[rgb]{0.25,0.50,0.50}{##1}}}
\expandafter\def\csname PY@tok@cs\endcsname{\let\PY@it=\textit\def\PY@tc##1{\textcolor[rgb]{0.25,0.50,0.50}{##1}}}

\def\PYZbs{\char`\\}
\def\PYZus{\char`\_}
\def\PYZob{\char`\{}
\def\PYZcb{\char`\}}
\def\PYZca{\char`\^}
\def\PYZam{\char`\&}
\def\PYZlt{\char`\<}
\def\PYZgt{\char`\>}
\def\PYZsh{\char`\#}
\def\PYZpc{\char`\%}
\def\PYZdl{\char`\$}
\def\PYZhy{\char`\-}
\def\PYZsq{\char`\'}
\def\PYZdq{\char`\"}
\def\PYZti{\char`\~}
% for compatibility with earlier versions
\def\PYZat{@}
\def\PYZlb{[}
\def\PYZrb{]}
\makeatother


    % Exact colors from NB
    \definecolor{incolor}{rgb}{0.0, 0.0, 0.5}
    \definecolor{outcolor}{rgb}{0.545, 0.0, 0.0}



    
    % Prevent overflowing lines due to hard-to-break entities
    \sloppy 
    % Setup hyperref package
    \hypersetup{
      breaklinks=true,  % so long urls are correctly broken across lines
      colorlinks=true,
      urlcolor=urlcolor,
      linkcolor=linkcolor,
      citecolor=citecolor,
      }
    % Slightly bigger margins than the latex defaults
    
    \geometry{verbose,tmargin=1in,bmargin=1in,lmargin=1in,rmargin=1in}
    
    

    \begin{document}
    
    
    \maketitle
    
    

    
    \section{Strategy Learner}\label{strategy-learner}

ML for trading Udacity Course exercise

More info: https://quantsoftware.gatech.edu/Strategy\_learner

A transcription of the Udacity Course lectures can be find on
https://docs.google.com/document/d/1ELqlnuTSdc9-MDHOkV0uvSY4RmI1eslyQlU9DgOY\_jc/edit?usp=sharing

Kairoart 2018 """

    \subsection{Overview}\label{overview}

In this project you will design a learning trading agent. You must draw
on the learners you have created so far in the course. Your choices are:

\begin{enumerate}
\def\labelenumi{\arabic{enumi}.}
\tightlist
\item
  Regression or classification-based learner: Create a strategy using
  your Random Forest learner. Suggestions if you follow this approach:
  Classification\_Trader\_Hints. Important note, if you choose this
  method, you must set the leaf\_size for your learner to 5 or greater.
  This is to avoid degenerate overfitting in-sample.
\item
  Reinforcement Learner-based approach: Create a Q-learning-based
  strategy using your Q-Learner. Read the Classification\_Trader\_Hints
  first, because many of the ideas there are relevant for the Q trader,
  then see Q\_Trader\_Hints
\item
  Optimization-based learner: Create a scan-based strategy using an
  optimizer. Read the Classification\_Trader\_Hints first, because many
  of the ideas there are relevant for the Opto trader, then see
  Opto\_Trader\_Hints
\end{enumerate}

Regardless of your choice above, your learner should work in the
following way:

\begin{itemize}
\tightlist
\item
  In the training phase (e.g., addEvidence()) your learner will be
  provided with a stock symbol and a time period. It should use this
  data to learn a strategy. For instance, for a regression-based learner
  it will use this data to make predictions about future price changes.
\item
  In the testing phase (e.g., testPolicy()) your learner will be
  provided a symbol and a date range. All learning should be turned OFF
  during this phase.
\end{itemize}

If the date range is the same as used for the training, it is an
in-sample test. Otherwise it is an out-of-sample test. Your learner
should return a trades dataframe like it did in the last project. Here
are some important requirements: Your testPolicy() method should be much
faster than your addEvidence() method. The timeout requirements (see
rubric) will be set accordingly. Multiple calls to your testPolicy()
method should return exactly the same result.

    \subsection{Tasks}\label{tasks}

\begin{itemize}
\tightlist
\item
  Devise numerical/technical indicators to evaluate the state of a stock
  on each day.
\item
  Build a strategy learner based on one of the learners described above
  that uses the indicators.
\item
  Test/debug the strategy learner on specific symbol/time period
  problems.
\item
  Write a report describing your learning strategy.
\end{itemize}

    \subsection{Data Details, Dates and
Rules}\label{data-details-dates-and-rules}

\begin{itemize}
\tightlist
\item
  For your report, trade only the symbol JPM. This will enable us to
  more easily compare results. We will test your learner with other
  symbols as well.
\item
  You may use data from other symbols (such as SPY) to inform your
  strategy.
\item
  The in sample/development period is January 1, 2008 to December 31
  2009.
\item
  The out of sample/testing period is January 1, 2010 to December 31
  2011.
\item
  Starting cash is 100,000.
\item
  Allowable positions are: 1000 shares long, 1000 shares short, 0
  shares.
\item
  Benchmark: The performance of a portfolio starting with 100,000 cash,
  investing in 1000 shares of the symbol in use and holding that
  position. Include transaction costs.
\item
  There is no limit on leverage.
\item
  Transaction costs: Commission will always be 0.00, Impact may vary,
  and will be passed in as a parameter to the learner.
\item
  Minimize use of herrings.
\end{itemize}

    \subsection{Implement Strategy
Learner}\label{implement-strategy-learner}

For this part of the project you should develop a learner that can learn
a trading policy using your learner. You should be able to use your
Q-Learner or RTLearner from the earlier project directly, with no
changes. If you want to use the optimization approach, you will need to
create new code or that. You will need to write code in
StrategyLearner.py to "wrap" your learner appropriately to frame the
trading problem for it.

    \subsubsection{StrategyLearner API}\label{strategylearner-api}

import StrategyLearner as sl learner = sl.StrategyLearner(verbose =
False, impact = 0.000) \# constructor learner.addEvidence(symbol =
"AAPL", sd=dt.datetime(2008,1,1), ed=dt.datetime(2009,12,31), sv =
100000) \# training phase df\_trades = learner.testPolicy(symbol =
"AAPL", sd=dt.datetime(2010,1,1), ed=dt.datetime(2011,12,31), sv =
100000) \# testing phase

The input parameters are:

\begin{itemize}
\tightlist
\item
  verbose: if False do not generate any output
\item
  impact: The market impact of each transaction.
\item
  symbol: the stock symbol to train on
\item
  sd: A datetime object that represents the start date
\item
  ed: A datetime object that represents the end date
\item
  sv: Start value of the portfolio
\end{itemize}

The output result is:

\begin{itemize}
\tightlist
\item
  df\_trades: A data frame whose values represent trades for each day.
  Legal values are +1000.0 indicating a BUY of 1000 shares, -1000.0
  indicating a SELL of 1000 shares, and 0.0 indicating NOTHING. Values
  of +2000 and -2000 for trades are also legal when switching from long
  to short or short to long so long as net holdings are constrained to
  -1000, 0, and 1000.
\end{itemize}

    \subsection{Goal}\label{goal}

Implement a StrategyLearner that trains a QLearner for trading a symbol.

    \subsection{Import libraries}\label{import-libraries}

    \begin{Verbatim}[commandchars=\\\{\}]
{\color{incolor}In [{\color{incolor}1}]:} \PY{k+kn}{import} \PY{n+nn}{pandas} \PY{k}{as} \PY{n+nn}{pd}
        \PY{k+kn}{import} \PY{n+nn}{numpy} \PY{k}{as} \PY{n+nn}{np}  
        \PY{k+kn}{import} \PY{n+nn}{datetime} \PY{k}{as} \PY{n+nn}{dt}
        
        \PY{k+kn}{from} \PY{n+nn}{util} \PY{k}{import} \PY{n}{create\PYZus{}df\PYZus{}benchmark}\PY{p}{,} \PY{n}{get\PYZus{}data}
        \PY{k+kn}{from} \PY{n+nn}{strategyLearner} \PY{k}{import} \PY{n}{strategyLearner}
        \PY{k+kn}{from} \PY{n+nn}{marketsim} \PY{k}{import} \PY{n}{compute\PYZus{}portvals\PYZus{}single\PYZus{}symbol}\PY{p}{,} \PY{n}{market\PYZus{}simulator}
        \PY{k+kn}{from} \PY{n+nn}{indicators} \PY{k}{import} \PY{n}{get\PYZus{}momentum}\PY{p}{,} \PY{n}{get\PYZus{}sma}\PY{p}{,} \PY{n}{get\PYZus{}sma\PYZus{}indicator}\PY{p}{,} \PY{n}{compute\PYZus{}bollinger\PYZus{}value}\PY{p}{,} \PY{n}{get\PYZus{}RSI}\PY{p}{,} \PY{n}{plot\PYZus{}cum\PYZus{}return}\PY{p}{,}  \PY{n}{plot\PYZus{}momentum}\PY{p}{,} \PY{n}{plot\PYZus{}sma\PYZus{}indicator}\PY{p}{,} \PY{n}{plot\PYZus{}rsi\PYZus{}indicator}\PY{p}{,} \PY{n}{plot\PYZus{}momentum\PYZus{}sma\PYZus{}indicator}
\end{Verbatim}


    
    
    
    
    \subsection{Initial Variables}\label{initial-variables}

    \begin{Verbatim}[commandchars=\\\{\}]
{\color{incolor}In [{\color{incolor}2}]:} \PY{n}{start\PYZus{}val} \PY{o}{=} \PY{l+m+mi}{100000}
        \PY{n}{symbol} \PY{o}{=} \PY{l+s+s2}{\PYZdq{}}\PY{l+s+s2}{JPM}\PY{l+s+s2}{\PYZdq{}}
        \PY{n}{commission} \PY{o}{=} \PY{l+m+mf}{0.00}
        \PY{n}{impact} \PY{o}{=} \PY{l+m+mf}{0.0}
        \PY{n}{num\PYZus{}shares} \PY{o}{=} \PY{l+m+mi}{1000}
\end{Verbatim}


    \subsection{In-sample performance}\label{in-sample-performance}

Show the performances of portfolio and benchmark in the in-sample
period.

    \begin{Verbatim}[commandchars=\\\{\}]
{\color{incolor}In [{\color{incolor}3}]:} \PY{c+c1}{\PYZsh{} Specify the start and end dates for this period.}
        \PY{n}{start\PYZus{}d} \PY{o}{=} \PY{n}{dt}\PY{o}{.}\PY{n}{datetime}\PY{p}{(}\PY{l+m+mi}{2008}\PY{p}{,} \PY{l+m+mi}{1}\PY{p}{,} \PY{l+m+mi}{1}\PY{p}{)}
        \PY{n}{end\PYZus{}d} \PY{o}{=} \PY{n}{dt}\PY{o}{.}\PY{n}{datetime}\PY{p}{(}\PY{l+m+mi}{2009}\PY{p}{,} \PY{l+m+mi}{12}\PY{p}{,} \PY{l+m+mi}{31}\PY{p}{)}
        
        \PY{c+c1}{\PYZsh{} Get benchmark data}
        \PY{n}{benchmark\PYZus{}prices} \PY{o}{=} \PY{n}{get\PYZus{}data}\PY{p}{(}\PY{p}{[}\PY{n}{symbol}\PY{p}{]}\PY{p}{,} \PY{n}{pd}\PY{o}{.}\PY{n}{date\PYZus{}range}\PY{p}{(}\PY{n}{start\PYZus{}d}\PY{p}{,} \PY{n}{end\PYZus{}d}\PY{p}{)}\PY{p}{,} \PY{n}{addSPY}\PY{o}{=}\PY{k+kc}{False}\PY{p}{)}\PY{o}{.}\PY{n}{dropna}\PY{p}{(}\PY{p}{)}
        
        \PY{c+c1}{\PYZsh{} Create benchmark data: Benchmark is a portfolio starting with \PYZdl{}100,000, investing in 1000 shares of symbol and holding that position}
        \PY{n}{df\PYZus{}benchmark\PYZus{}trades} \PY{o}{=} \PY{n}{create\PYZus{}df\PYZus{}benchmark}\PY{p}{(}\PY{n}{symbol}\PY{p}{,} \PY{n}{start\PYZus{}d}\PY{p}{,} \PY{n}{end\PYZus{}d}\PY{p}{,} \PY{n}{num\PYZus{}shares}\PY{p}{)}
        
        \PY{c+c1}{\PYZsh{}print (df\PYZus{}benchmark\PYZus{}trades)}
        
        \PY{c+c1}{\PYZsh{} Train and test a StrategyLearner}
        \PY{c+c1}{\PYZsh{} Set verbose to True will print out and plot the cumulative return for each training epoch}
        \PY{n}{stl} \PY{o}{=} \PY{n}{strategyLearner}\PY{p}{(}\PY{n}{num\PYZus{}shares}\PY{o}{=}\PY{n}{num\PYZus{}shares}\PY{p}{,} \PY{n}{impact}\PY{o}{=}\PY{n}{impact}\PY{p}{,} 
                              \PY{n}{commission}\PY{o}{=}\PY{n}{commission}\PY{p}{,} \PY{n}{verbose}\PY{o}{=}\PY{k+kc}{True}\PY{p}{,}
                              \PY{n}{num\PYZus{}states}\PY{o}{=}\PY{l+m+mi}{3000}\PY{p}{,} \PY{n}{num\PYZus{}actions}\PY{o}{=}\PY{l+m+mi}{3}\PY{p}{)}
        \PY{n}{stl}\PY{o}{.}\PY{n}{add\PYZus{}evidence}\PY{p}{(}\PY{n}{symbol}\PY{o}{=}\PY{n}{symbol}\PY{p}{,} \PY{n}{start\PYZus{}val}\PY{o}{=}\PY{n}{start\PYZus{}val}\PY{p}{,} 
                         \PY{n}{start\PYZus{}date}\PY{o}{=}\PY{n}{start\PYZus{}d}\PY{p}{,} \PY{n}{end\PYZus{}date}\PY{o}{=}\PY{n}{end\PYZus{}d}\PY{p}{)}
        \PY{n}{df\PYZus{}trades} \PY{o}{=} \PY{n}{stl}\PY{o}{.}\PY{n}{test\PYZus{}policy}\PY{p}{(}\PY{n}{symbol}\PY{o}{=}\PY{n}{symbol}\PY{p}{,} \PY{n}{start\PYZus{}date}\PY{o}{=}\PY{n}{start\PYZus{}d}\PY{p}{,}
                                    \PY{n}{end\PYZus{}date}\PY{o}{=}\PY{n}{end\PYZus{}d}\PY{p}{)}
        \PY{c+c1}{\PYZsh{}print (df\PYZus{}trades)}
\end{Verbatim}


    \begin{Verbatim}[commandchars=\\\{\}]
/home/emi/Jupyter/ML4T\_2018Spring/strategy\_learner/indicators.py:100: FutureWarning:

pd.rolling\_apply is deprecated for Series and will be removed in a future version, replace with 
	Series.rolling(window=10,center=False).apply(func=<function>,args=<tuple>,kwargs=<dict>)


    \end{Verbatim}

    \begin{Verbatim}[commandchars=\\\{\}]
1 -0.15459999999999996
2 -0.13849999999999996
3 0.12729999999999997
4 0.29869999999999997
5 0.4162999999999999
6 0.4162999999999999
7 0.4162999999999999
8 0.4162999999999999
9 0.4162999999999999
10 0.4162999999999999
11 0.4162999999999999
12 0.4162999999999999
13 0.4162999999999999
14 0.4162999999999999
15 0.4162999999999999
16 0.4162999999999999
17 0.4162999999999999
18 0.4162999999999999
19 0.4162999999999999
20 0.4162999999999999
21 0.4162999999999999

    \end{Verbatim}

    \begin{center}
    \adjustimage{max size={0.9\linewidth}{0.9\paperheight}}{output_12_2.png}
    \end{center}
    { \hspace*{\fill} \\}
    
    \subsubsection{Training market
simulator}\label{training-market-simulator}

    \begin{Verbatim}[commandchars=\\\{\}]
{\color{incolor}In [{\color{incolor}4}]:} \PY{c+c1}{\PYZsh{} Retrieve performance stats via a market simulator}
        \PY{n+nb}{print} \PY{p}{(}\PY{l+s+s2}{\PYZdq{}}\PY{l+s+s2}{Performances during training period for }\PY{l+s+si}{\PYZob{}\PYZcb{}}\PY{l+s+s2}{\PYZdq{}}\PY{o}{.}\PY{n}{format}\PY{p}{(}\PY{n}{symbol}\PY{p}{)}\PY{p}{)}
        \PY{n+nb}{print} \PY{p}{(}\PY{l+s+s2}{\PYZdq{}}\PY{l+s+s2}{Date Range: }\PY{l+s+si}{\PYZob{}\PYZcb{}}\PY{l+s+s2}{ to }\PY{l+s+si}{\PYZob{}\PYZcb{}}\PY{l+s+s2}{\PYZdq{}}\PY{o}{.}\PY{n}{format}\PY{p}{(}\PY{n}{start\PYZus{}d}\PY{p}{,} \PY{n}{end\PYZus{}d}\PY{p}{)}\PY{p}{)}
        \PY{n}{orders\PYZus{}count}\PY{p}{,} \PY{n}{sharpe\PYZus{}ratio}\PY{p}{,} \PY{n}{cum\PYZus{}ret}\PY{p}{,} \PY{n}{std\PYZus{}daily\PYZus{}ret}\PY{p}{,} \PY{n}{avg\PYZus{}daily\PYZus{}ret}\PY{p}{,} \PY{n}{final\PYZus{}value} \PY{o}{=} \PY{n}{market\PYZus{}simulator}\PY{p}{(}\PY{n}{df\PYZus{}trades}\PY{p}{,} \PY{n}{df\PYZus{}benchmark\PYZus{}trades}\PY{p}{,} \PY{n}{symbol}\PY{o}{=}\PY{n}{symbol}\PY{p}{,} 
                         \PY{n}{start\PYZus{}val}\PY{o}{=}\PY{n}{start\PYZus{}val}\PY{p}{,} \PY{n}{commission}\PY{o}{=}\PY{n}{commission}\PY{p}{,} \PY{n}{impact}\PY{o}{=}\PY{n}{impact}\PY{p}{,} \PY{n}{title}\PY{o}{=}\PY{l+s+s2}{\PYZdq{}}\PY{l+s+s2}{Portfolio Value}\PY{l+s+s2}{\PYZdq{}}\PY{p}{,} \PY{n}{xtitle}\PY{o}{=}\PY{l+s+s2}{\PYZdq{}}\PY{l+s+s2}{Dates}\PY{l+s+s2}{\PYZdq{}}\PY{p}{,} \PY{n}{ytitle}\PY{o}{=}\PY{l+s+s2}{\PYZdq{}}\PY{l+s+s2}{Value}\PY{l+s+s2}{\PYZdq{}}\PY{p}{)}
\end{Verbatim}


    \begin{Verbatim}[commandchars=\\\{\}]
Performances during training period for JPM
Date Range: 2008-01-01 00:00:00 to 2009-12-31 00:00:00
Sharpe Ratio of Portfolio: 1.4565263293302735
Sharpe Ratio of Benchmark : 0.15691840642403027

Cumulative Return of Portfolio: 0.4162999999999999
Cumulative Return of Benchmark : 0.012299999999999978

Standard Deviation of Portfolio: 0.008049350943393187
Standard Deviation of Benchmark : 0.017004366271213767

Average Daily Return of Portfolio: 0.0007385483494512656
Average Daily Return of Benchmark : 0.00016808697819094035

Final Portfolio Value: 141630.0
Final Benchmark Value: 101230.0

Portfolio Orders count: 494

    \end{Verbatim}

    \begin{Verbatim}[commandchars=\\\{\}]
/home/emi/anaconda3/lib/python3.6/site-packages/plotly/graph\_objs/\_deprecations.py:396: DeprecationWarning:

plotly.graph\_objs.Margin is deprecated.
Please replace it with one of the following more specific types
  - plotly.graph\_objs.layout.Margin



    \end{Verbatim}

    \begin{center}
    \adjustimage{max size={0.9\linewidth}{0.9\paperheight}}{output_14_2.png}
    \end{center}
    { \hspace*{\fill} \\}
    
    \subsection{Out of sample performance}\label{out-of-sample-performance}

Show the performances of portfolio and benchmark in the out of sample
period. Use the same StrategyLearner trained above and retrieve a trades
dataframe via test\_policy.

    \begin{Verbatim}[commandchars=\\\{\}]
{\color{incolor}In [{\color{incolor}5}]:} \PY{c+c1}{\PYZsh{} Specify the start and end dates for this period.}
        \PY{n}{start\PYZus{}d} \PY{o}{=} \PY{n}{dt}\PY{o}{.}\PY{n}{datetime}\PY{p}{(}\PY{l+m+mi}{2010}\PY{p}{,} \PY{l+m+mi}{1}\PY{p}{,} \PY{l+m+mi}{1}\PY{p}{)}
        \PY{n}{end\PYZus{}d} \PY{o}{=} \PY{n}{dt}\PY{o}{.}\PY{n}{datetime}\PY{p}{(}\PY{l+m+mi}{2011}\PY{p}{,} \PY{l+m+mi}{12}\PY{p}{,} \PY{l+m+mi}{31}\PY{p}{)}
        
        \PY{c+c1}{\PYZsh{} Get benchmark data}
        \PY{n}{benchmark\PYZus{}prices} \PY{o}{=} \PY{n}{get\PYZus{}data}\PY{p}{(}\PY{p}{[}\PY{n}{symbol}\PY{p}{]}\PY{p}{,} \PY{n}{pd}\PY{o}{.}\PY{n}{date\PYZus{}range}\PY{p}{(}\PY{n}{start\PYZus{}d}\PY{p}{,} \PY{n}{end\PYZus{}d}\PY{p}{)}\PY{p}{,} \PY{n}{addSPY}\PY{o}{=}\PY{k+kc}{False}\PY{p}{)}\PY{o}{.}\PY{n}{dropna}\PY{p}{(}\PY{p}{)}
        
        \PY{c+c1}{\PYZsh{} Create benchmark data: Benchmark is a portfolio starting with \PYZdl{}100,000, investing in 1000 shares of symbol and holding that position}
        \PY{n}{df\PYZus{}benchmark\PYZus{}trades} \PY{o}{=} \PY{n}{create\PYZus{}df\PYZus{}benchmark}\PY{p}{(}\PY{n}{symbol}\PY{p}{,} \PY{n}{start\PYZus{}d}\PY{p}{,} \PY{n}{end\PYZus{}d}\PY{p}{,} \PY{n}{num\PYZus{}shares}\PY{p}{)}
        
        \PY{c+c1}{\PYZsh{}print (df\PYZus{}benchmark\PYZus{}trades)}
        
        \PY{c+c1}{\PYZsh{} Test a StrategyLearner}
        \PY{c+c1}{\PYZsh{} Use the same StrategyLearner trained above and retrieve a trades dataframe via test\PYZus{}policy}
        \PY{n}{df\PYZus{}trades} \PY{o}{=} \PY{n}{stl}\PY{o}{.}\PY{n}{test\PYZus{}policy}\PY{p}{(}\PY{n}{symbol}\PY{o}{=}\PY{n}{symbol}\PY{p}{,} \PY{n}{start\PYZus{}date}\PY{o}{=}\PY{n}{start\PYZus{}d}\PY{p}{,}
                                    \PY{n}{end\PYZus{}date}\PY{o}{=}\PY{n}{end\PYZus{}d}\PY{p}{)}
        \PY{c+c1}{\PYZsh{}print (df\PYZus{}trades)}
\end{Verbatim}


    \begin{Verbatim}[commandchars=\\\{\}]
/home/emi/Jupyter/ML4T\_2018Spring/strategy\_learner/indicators.py:100: FutureWarning:

pd.rolling\_apply is deprecated for Series and will be removed in a future version, replace with 
	Series.rolling(window=10,center=False).apply(func=<function>,args=<tuple>,kwargs=<dict>)


    \end{Verbatim}

    \subsubsection{Test market simulator}\label{test-market-simulator}

    \begin{Verbatim}[commandchars=\\\{\}]
{\color{incolor}In [{\color{incolor}6}]:} \PY{c+c1}{\PYZsh{} Retrieve performance stats via a market simulator}
        \PY{n+nb}{print} \PY{p}{(}\PY{l+s+s2}{\PYZdq{}}\PY{l+s+s2}{Performances during test period for }\PY{l+s+si}{\PYZob{}\PYZcb{}}\PY{l+s+s2}{\PYZdq{}}\PY{o}{.}\PY{n}{format}\PY{p}{(}\PY{n}{symbol}\PY{p}{)}\PY{p}{)}
        \PY{n+nb}{print} \PY{p}{(}\PY{l+s+s2}{\PYZdq{}}\PY{l+s+s2}{Date Range: }\PY{l+s+si}{\PYZob{}\PYZcb{}}\PY{l+s+s2}{ to }\PY{l+s+si}{\PYZob{}\PYZcb{}}\PY{l+s+s2}{\PYZdq{}}\PY{o}{.}\PY{n}{format}\PY{p}{(}\PY{n}{start\PYZus{}d}\PY{p}{,} \PY{n}{end\PYZus{}d}\PY{p}{)}\PY{p}{)}
        \PY{n}{orders\PYZus{}count}\PY{p}{,} \PY{n}{sharpe\PYZus{}ratio}\PY{p}{,} \PY{n}{cum\PYZus{}ret}\PY{p}{,} \PY{n}{std\PYZus{}daily\PYZus{}ret}\PY{p}{,} \PY{n}{avg\PYZus{}daily\PYZus{}ret}\PY{p}{,} \PY{n}{final\PYZus{}value} \PY{o}{=} \PY{n}{market\PYZus{}simulator}\PY{p}{(}\PY{n}{df\PYZus{}trades}\PY{p}{,} \PY{n}{df\PYZus{}benchmark\PYZus{}trades}\PY{p}{,} \PY{n}{symbol}\PY{o}{=}\PY{n}{symbol}\PY{p}{,} 
                         \PY{n}{start\PYZus{}val}\PY{o}{=}\PY{n}{start\PYZus{}val}\PY{p}{,} \PY{n}{commission}\PY{o}{=}\PY{n}{commission}\PY{p}{,} \PY{n}{impact}\PY{o}{=}\PY{n}{impact}\PY{p}{,} \PY{n}{title}\PY{o}{=}\PY{l+s+s2}{\PYZdq{}}\PY{l+s+s2}{Portfolio Value}\PY{l+s+s2}{\PYZdq{}}\PY{p}{,} \PY{n}{xtitle}\PY{o}{=}\PY{l+s+s2}{\PYZdq{}}\PY{l+s+s2}{Dates}\PY{l+s+s2}{\PYZdq{}}\PY{p}{,} \PY{n}{ytitle}\PY{o}{=}\PY{l+s+s2}{\PYZdq{}}\PY{l+s+s2}{Value}\PY{l+s+s2}{\PYZdq{}}\PY{p}{)}
\end{Verbatim}


    \begin{Verbatim}[commandchars=\\\{\}]
Performances during test period for JPM
Date Range: 2010-01-01 00:00:00 to 2011-12-31 00:00:00
Sharpe Ratio of Portfolio: 0.06012262689818714
Sharpe Ratio of Benchmark : -0.2568129607376273

Cumulative Return of Portfolio: 0.0020999999999999908
Cumulative Return of Benchmark : -0.08340000000000003

Standard Deviation of Portfolio: 0.006219177700433581
Standard Deviation of Benchmark : 0.008481007498803986

Average Daily Return of Portfolio: 2.3554323928895607e-05
Average Daily Return of Benchmark : -0.00013720316019481813

Final Portfolio Value: 100210.0
Final Benchmark Value: 91660.0

Portfolio Orders count: 493

    \end{Verbatim}

    \begin{Verbatim}[commandchars=\\\{\}]
/home/emi/anaconda3/lib/python3.6/site-packages/plotly/graph\_objs/\_deprecations.py:396: DeprecationWarning:

plotly.graph\_objs.Margin is deprecated.
Please replace it with one of the following more specific types
  - plotly.graph\_objs.layout.Margin



    \end{Verbatim}

    \begin{center}
    \adjustimage{max size={0.9\linewidth}{0.9\paperheight}}{output_18_2.png}
    \end{center}
    { \hspace*{\fill} \\}
    
    \begin{Verbatim}[commandchars=\\\{\}]
{\color{incolor}In [{\color{incolor}7}]:} \PY{n}{df\PYZus{}trades}\PY{o}{.}\PY{n}{info}\PY{p}{(}\PY{p}{)}
\end{Verbatim}


    \begin{Verbatim}[commandchars=\\\{\}]
<class 'pandas.core.frame.DataFrame'>
DatetimeIndex: 118 entries, 2010-01-19 to 2011-12-29
Data columns (total 1 columns):
Shares    118 non-null int64
dtypes: int64(1)
memory usage: 1.8 KB

    \end{Verbatim}

    \section{Report}\label{report}

Describe the steps you took to frame the trading problem as a learning
problem for your learner. What are your indicators? Did you adjust the
data in any way (dicretization, standardization)? Why or why not?

\textbf{Steps:}

\begin{enumerate}
\def\labelenumi{\arabic{enumi}.}
\tightlist
\item
  Get adjusted close prices for symbol.
\item
  Set the QLearner parameters:

  \begin{itemize}
  \tightlist
  \item
    num\_shares: The number of shares that can be traded in one order
  \item
    epochs: The number of times to train the QLearner
  \item
    num\_steps: The number of steps used in getting thresholds for the
    discretization process. It is the number of groups to put data into.
  \end{itemize}
\item
  Compute technical indicators and use them as features to be fed into a
  Q-learner.

  \begin{itemize}
  \tightlist
  \item
    Momentum
  \item
    SMA indicator
  \item
    RSI indicator
  \end{itemize}
\item
  Get features and thresholds.
\item
  Define states, actions, and rewards.

  \begin{itemize}
  \tightlist
  \item
    States are combinations of our features.
  \item
    Actions are buy, sell, do nothing.
  \item
    Rewards: Calculate the daily reward as a percentage change in
    prices:

    \begin{itemize}
    \tightlist
    \item
      Position is long: if the price goes up (curr\_price \textgreater{}
      prev\_price), we get a positive reward; otherwise, we get a
      negative reward
    \item
      Position is short: if the price goes down, we get a positive
      reward; otherwise, we a negative reward
    \item
      Position is cash: we get no reward
    \end{itemize}
  \end{itemize}
\item
  Set initial position holding to nothing.
\item
  Create a series that captures order signals based on actions taken.
\item
  Iterate over the data by date.

  \begin{itemize}
  \tightlist
  \item
    Discretize features and return a state. Get a state; add 1 to
    position so that states \textgreater{}= 0.
  \item
    On the first day, get an action without updating the Q-table.
  \item
    On the last day, close any open positions.
  \item
    Add new\_pos to orders.
  \item
    Update current position.
  \end{itemize}
\item
  Create a trade dataframe.
\item
  Training: Choose the training period and you iterate over that
  training period and update your Q-table on each iteration. When you
  reach the end of that training period you backtest to see how good the
  model is and you go back and repeat, until the model quits getting
  better. Once it's converged you stop, you've got your model.
\item
  Testing the model: You just backtest it on later data.
\end{enumerate}

    \subsection{Indicator charts}\label{indicator-charts}

    \subsubsection{Momentum chart}\label{momentum-chart}

    \begin{Verbatim}[commandchars=\\\{\}]
{\color{incolor}In [{\color{incolor}13}]:} \PY{c+c1}{\PYZsh{} Specify the start and end dates for this period.}
         \PY{n}{start\PYZus{}d} \PY{o}{=} \PY{n}{dt}\PY{o}{.}\PY{n}{datetime}\PY{p}{(}\PY{l+m+mi}{2008}\PY{p}{,} \PY{l+m+mi}{1}\PY{p}{,} \PY{l+m+mi}{1}\PY{p}{)}
         \PY{n}{end\PYZus{}d} \PY{o}{=} \PY{n}{dt}\PY{o}{.}\PY{n}{datetime}\PY{p}{(}\PY{l+m+mi}{2009}\PY{p}{,} \PY{l+m+mi}{12}\PY{p}{,} \PY{l+m+mi}{31}\PY{p}{)}
         
         \PY{c+c1}{\PYZsh{} Set dates}
         \PY{n}{dates} \PY{o}{=} \PY{n}{pd}\PY{o}{.}\PY{n}{date\PYZus{}range}\PY{p}{(}\PY{n}{start\PYZus{}d}\PY{p}{,} \PY{n}{end\PYZus{}d}\PY{p}{)}
         
         \PY{c+c1}{\PYZsh{} Get adjusted close prices for symbol}
         \PY{n}{df} \PY{o}{=} \PY{n}{get\PYZus{}data}\PY{p}{(}\PY{p}{[}\PY{n}{symbol}\PY{p}{]}\PY{p}{,} \PY{n}{dates}\PY{p}{,} \PY{n}{addSPY}\PY{o}{=}\PY{k+kc}{False}\PY{p}{)}
         \PY{n}{df} \PY{o}{=} \PY{n}{df}\PY{o}{.}\PY{n}{dropna}\PY{p}{(}\PY{p}{)}
         
         
         \PY{c+c1}{\PYZsh{} Normalize the prices Dataframe}
         \PY{n}{normed} \PY{o}{=} \PY{n}{pd}\PY{o}{.}\PY{n}{DataFrame}\PY{p}{(}\PY{p}{)}
         \PY{k}{for} \PY{n}{column} \PY{o+ow}{in} \PY{n}{df}\PY{p}{:}
             \PY{n}{normed}\PY{p}{[}\PY{n}{column}\PY{p}{]} \PY{o}{=} \PY{n}{df}\PY{p}{[}\PY{n}{column}\PY{p}{]}\PY{o}{.}\PY{n}{values} \PY{o}{/} \PY{n}{df}\PY{p}{[}\PY{n}{column}\PY{p}{]}\PY{o}{.}\PY{n}{iloc}\PY{p}{[}\PY{l+m+mi}{0}\PY{p}{]}\PY{p}{;}
         
         \PY{c+c1}{\PYZsh{} 2. Compute momentum}
         \PY{n}{sym\PYZus{}mom} \PY{o}{=} \PY{n}{get\PYZus{}momentum}\PY{p}{(}\PY{n}{normed}\PY{p}{[}\PY{n}{column}\PY{p}{]}\PY{p}{,} \PY{n}{window}\PY{o}{=}\PY{l+m+mi}{10}\PY{p}{)}
         
         \PY{c+c1}{\PYZsh{} 3. Plot raw JPM values and Momentum}
         \PY{n}{plot\PYZus{}momentum}\PY{p}{(}\PY{n}{df}\PY{o}{.}\PY{n}{index}\PY{p}{,} \PY{n}{normed}\PY{p}{[}\PY{n}{column}\PY{p}{]}\PY{p}{,} \PY{n}{sym\PYZus{}mom}\PY{p}{,} \PY{l+s+s2}{\PYZdq{}}\PY{l+s+s2}{Momentum Indicator}\PY{l+s+s2}{\PYZdq{}}\PY{p}{,} \PY{p}{(}\PY{l+m+mi}{12}\PY{p}{,} \PY{l+m+mi}{8}\PY{p}{)}\PY{p}{)}
\end{Verbatim}


    \begin{center}
    \adjustimage{max size={0.9\linewidth}{0.9\paperheight}}{output_23_0.png}
    \end{center}
    { \hspace*{\fill} \\}
    
    \subsubsection{SMA indicator}\label{sma-indicator}

    \begin{Verbatim}[commandchars=\\\{\}]
{\color{incolor}In [{\color{incolor}9}]:} \PY{c+c1}{\PYZsh{} Specify the start and end dates for this period.}
        \PY{n}{start\PYZus{}d} \PY{o}{=} \PY{n}{dt}\PY{o}{.}\PY{n}{datetime}\PY{p}{(}\PY{l+m+mi}{2008}\PY{p}{,} \PY{l+m+mi}{1}\PY{p}{,} \PY{l+m+mi}{1}\PY{p}{)}
        \PY{n}{end\PYZus{}d} \PY{o}{=} \PY{n}{dt}\PY{o}{.}\PY{n}{datetime}\PY{p}{(}\PY{l+m+mi}{2009}\PY{p}{,} \PY{l+m+mi}{12}\PY{p}{,} \PY{l+m+mi}{31}\PY{p}{)}
        
        \PY{c+c1}{\PYZsh{} Set dates}
        \PY{n}{dates} \PY{o}{=} \PY{n}{pd}\PY{o}{.}\PY{n}{date\PYZus{}range}\PY{p}{(}\PY{n}{start\PYZus{}d}\PY{p}{,} \PY{n}{end\PYZus{}d}\PY{p}{)}
        
        \PY{c+c1}{\PYZsh{} Get adjusted close prices for symbol}
        \PY{n}{df} \PY{o}{=} \PY{n}{get\PYZus{}data}\PY{p}{(}\PY{p}{[}\PY{n}{symbol}\PY{p}{]}\PY{p}{,} \PY{n}{dates}\PY{p}{,} \PY{n}{addSPY}\PY{o}{=}\PY{k+kc}{False}\PY{p}{)}
        \PY{n}{df} \PY{o}{=} \PY{n}{df}\PY{o}{.}\PY{n}{dropna}\PY{p}{(}\PY{p}{)}
        
        \PY{c+c1}{\PYZsh{} Normalize the prices Dataframe}
        \PY{n}{normed} \PY{o}{=} \PY{n}{pd}\PY{o}{.}\PY{n}{DataFrame}\PY{p}{(}\PY{p}{)}
        \PY{k}{for} \PY{n}{column} \PY{o+ow}{in} \PY{n}{df}\PY{p}{:}
            \PY{n}{normed}\PY{p}{[}\PY{n}{column}\PY{p}{]} \PY{o}{=} \PY{n}{df}\PY{p}{[}\PY{n}{column}\PY{p}{]}\PY{o}{.}\PY{n}{values} \PY{o}{/} \PY{n}{df}\PY{p}{[}\PY{n}{column}\PY{p}{]}\PY{o}{.}\PY{n}{iloc}\PY{p}{[}\PY{l+m+mi}{0}\PY{p}{]}\PY{p}{;}
        
        \PY{c+c1}{\PYZsh{} Compute SMA}
        \PY{n}{sma\PYZus{}JPM}\PY{p}{,} \PY{n}{q} \PY{o}{=} \PY{n}{get\PYZus{}sma}\PY{p}{(}\PY{n}{normed}\PY{p}{[}\PY{n}{column}\PY{p}{]}\PY{p}{,} \PY{n}{window}\PY{o}{=}\PY{l+m+mi}{10}\PY{p}{)}
        
        \PY{c+c1}{\PYZsh{} Plot symbol values, SMA and SMA quality}
        \PY{n}{plot\PYZus{}sma\PYZus{}indicator}\PY{p}{(}\PY{n}{dates}\PY{p}{,} \PY{n}{df}\PY{o}{.}\PY{n}{index}\PY{p}{,} \PY{n}{normed}\PY{p}{[}\PY{n}{column}\PY{p}{]}\PY{p}{,} \PY{n}{sma\PYZus{}JPM}\PY{p}{,} \PY{n}{q}\PY{p}{,} \PY{l+s+s2}{\PYZdq{}}\PY{l+s+s2}{Simple Moving Average (SMA)}\PY{l+s+s2}{\PYZdq{}}\PY{p}{)}
\end{Verbatim}


    \begin{center}
    \adjustimage{max size={0.9\linewidth}{0.9\paperheight}}{output_25_0.png}
    \end{center}
    { \hspace*{\fill} \\}
    
    \subsubsection{Momentum/SMA cross
indicator}\label{momentumsma-cross-indicator}

    \begin{Verbatim}[commandchars=\\\{\}]
{\color{incolor}In [{\color{incolor}10}]:} \PY{c+c1}{\PYZsh{} Specify the start and end dates for this period.}
         \PY{n}{start\PYZus{}d} \PY{o}{=} \PY{n}{dt}\PY{o}{.}\PY{n}{datetime}\PY{p}{(}\PY{l+m+mi}{2008}\PY{p}{,} \PY{l+m+mi}{1}\PY{p}{,} \PY{l+m+mi}{1}\PY{p}{)}
         \PY{n}{end\PYZus{}d} \PY{o}{=} \PY{n}{dt}\PY{o}{.}\PY{n}{datetime}\PY{p}{(}\PY{l+m+mi}{2009}\PY{p}{,} \PY{l+m+mi}{12}\PY{p}{,} \PY{l+m+mi}{31}\PY{p}{)}
         
         \PY{c+c1}{\PYZsh{} Set dates}
         \PY{n}{dates} \PY{o}{=} \PY{n}{pd}\PY{o}{.}\PY{n}{date\PYZus{}range}\PY{p}{(}\PY{n}{start\PYZus{}d}\PY{p}{,} \PY{n}{end\PYZus{}d}\PY{p}{)}
         
         \PY{c+c1}{\PYZsh{} Get adjusted close prices for symbol}
         \PY{n}{df} \PY{o}{=} \PY{n}{get\PYZus{}data}\PY{p}{(}\PY{p}{[}\PY{n}{symbol}\PY{p}{]}\PY{p}{,} \PY{n}{dates}\PY{p}{,} \PY{n}{addSPY}\PY{o}{=}\PY{k+kc}{False}\PY{p}{)}
         \PY{n}{df} \PY{o}{=} \PY{n}{df}\PY{o}{.}\PY{n}{dropna}\PY{p}{(}\PY{p}{)}
         
         \PY{c+c1}{\PYZsh{} Compute momentum}
         \PY{n}{sym\PYZus{}mom} \PY{o}{=} \PY{n}{get\PYZus{}momentum}\PY{p}{(}\PY{n}{normed}\PY{p}{[}\PY{n}{column}\PY{p}{]}\PY{p}{,} \PY{n}{window}\PY{o}{=}\PY{l+m+mi}{10}\PY{p}{)}
         
         \PY{c+c1}{\PYZsh{} Compute SMA}
         \PY{n}{sma\PYZus{}JPM}\PY{p}{,} \PY{n}{q} \PY{o}{=} \PY{n}{get\PYZus{}sma}\PY{p}{(}\PY{n}{normed}\PY{p}{[}\PY{n}{column}\PY{p}{]}\PY{p}{,} \PY{n}{window}\PY{o}{=}\PY{l+m+mi}{10}\PY{p}{)}
         
         \PY{c+c1}{\PYZsh{} Plot symbol values, SMA and Momentum}
         \PY{n}{plot\PYZus{}momentum\PYZus{}sma\PYZus{}indicator}\PY{p}{(}\PY{n}{dates}\PY{p}{,} \PY{n}{df}\PY{o}{.}\PY{n}{index}\PY{p}{,} \PY{n}{normed}\PY{p}{[}\PY{n}{column}\PY{p}{]}\PY{p}{,} \PY{n}{sma\PYZus{}JPM}\PY{p}{,} \PY{n}{sym\PYZus{}mom}\PY{p}{,} \PY{l+s+s2}{\PYZdq{}}\PY{l+s+s2}{Momentum/SMA}\PY{l+s+s2}{\PYZdq{}}\PY{p}{)}
\end{Verbatim}


    \begin{center}
    \adjustimage{max size={0.9\linewidth}{0.9\paperheight}}{output_27_0.png}
    \end{center}
    { \hspace*{\fill} \\}
    
    \subsubsection{RSI indicator}\label{rsi-indicator}

    \begin{Verbatim}[commandchars=\\\{\}]
{\color{incolor}In [{\color{incolor}12}]:} \PY{c+c1}{\PYZsh{} Specify the start and end dates for this period.}
         \PY{n}{start\PYZus{}d} \PY{o}{=} \PY{n}{dt}\PY{o}{.}\PY{n}{datetime}\PY{p}{(}\PY{l+m+mi}{2008}\PY{p}{,} \PY{l+m+mi}{1}\PY{p}{,} \PY{l+m+mi}{1}\PY{p}{)}
         \PY{n}{end\PYZus{}d} \PY{o}{=} \PY{n}{dt}\PY{o}{.}\PY{n}{datetime}\PY{p}{(}\PY{l+m+mi}{2009}\PY{p}{,} \PY{l+m+mi}{12}\PY{p}{,} \PY{l+m+mi}{31}\PY{p}{)}
         
         \PY{c+c1}{\PYZsh{} Set dates}
         \PY{n}{dates} \PY{o}{=} \PY{n}{pd}\PY{o}{.}\PY{n}{date\PYZus{}range}\PY{p}{(}\PY{n}{start\PYZus{}d}\PY{p}{,} \PY{n}{end\PYZus{}d}\PY{p}{)}
         
         \PY{c+c1}{\PYZsh{} Get adjusted close prices for symbol}
         \PY{n}{df} \PY{o}{=} \PY{n}{get\PYZus{}data}\PY{p}{(}\PY{p}{[}\PY{n}{symbol}\PY{p}{]}\PY{p}{,} \PY{n}{dates}\PY{p}{,} \PY{n}{addSPY}\PY{o}{=}\PY{k+kc}{False}\PY{p}{)}
         \PY{n}{df} \PY{o}{=} \PY{n}{df}\PY{o}{.}\PY{n}{dropna}\PY{p}{(}\PY{p}{)}
         
         \PY{c+c1}{\PYZsh{} 1. Compute RSI}
         \PY{n}{rsi\PYZus{}JPM} \PY{o}{=} \PY{n}{get\PYZus{}RSI}\PY{p}{(}\PY{n}{df}\PY{p}{[}\PY{l+s+s1}{\PYZsq{}}\PY{l+s+s1}{JPM}\PY{l+s+s1}{\PYZsq{}}\PY{p}{]}\PY{p}{)}
         
         \PY{c+c1}{\PYZsh{} 2. Plot RSI}
         \PY{n}{plot\PYZus{}rsi\PYZus{}indicator}\PY{p}{(}\PY{n}{dates}\PY{p}{,} \PY{n}{df}\PY{o}{.}\PY{n}{index}\PY{p}{,} \PY{n}{df}\PY{p}{[}\PY{l+s+s1}{\PYZsq{}}\PY{l+s+s1}{JPM}\PY{l+s+s1}{\PYZsq{}}\PY{p}{]}\PY{p}{,} \PY{n}{rsi\PYZus{}JPM}\PY{p}{,} \PY{n}{window}\PY{o}{=}\PY{l+m+mi}{14}\PY{p}{,} 
                            \PY{n}{title}\PY{o}{=}\PY{l+s+s2}{\PYZdq{}}\PY{l+s+s2}{RSI Indicator}\PY{l+s+s2}{\PYZdq{}}\PY{p}{,} \PY{n}{fig\PYZus{}size}\PY{o}{=}\PY{p}{(}\PY{l+m+mi}{12}\PY{p}{,} \PY{l+m+mi}{6}\PY{p}{)}\PY{p}{)}
\end{Verbatim}


    \begin{Verbatim}[commandchars=\\\{\}]
/home/emi/Jupyter/ML4T\_2018Spring/strategy\_learner/indicators.py:100: FutureWarning:

pd.rolling\_apply is deprecated for Series and will be removed in a future version, replace with 
	Series.rolling(window=14,center=False).apply(func=<function>,args=<tuple>,kwargs=<dict>)


    \end{Verbatim}

    \begin{center}
    \adjustimage{max size={0.9\linewidth}{0.9\paperheight}}{output_29_1.png}
    \end{center}
    { \hspace*{\fill} \\}
    
    \section{TODO}\label{todo}

Write a PDF report describing your system. The centerpiece of your
report should be the description of how you utilized your learner to
determine trades


    % Add a bibliography block to the postdoc
    
    
    
    \end{document}
